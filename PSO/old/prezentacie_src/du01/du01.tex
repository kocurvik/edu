\documentclass{beamer}
%\usetheme{Ilmenau}
%\usecolortheme{beaver}

\usepackage[slovak,american]{babel}
\usepackage[utf8]{inputenc}
\usepackage{graphicx}
\usepackage{adjustbox}
 \usepackage{xcolor}
 
 \newsavebox\MBox
\newcommand\Cline[2][red]{{\sbox\MBox{$#2$}%
  \rlap{\usebox\MBox}\color{#1}\rule[-2.2\dp\MBox]{\wd\MBox}{1pt}}}

%\usefonttheme{serif}

%\definecolor{UKOrange}{HTML}{ef9424} %
\definecolor{UKOrange}{HTML}{7a2c18} %
\definecolor{UKBrown}{HTML}{a96d5e} %
\definecolor{UKLight}{HTML}{d8b6ab} %
\definecolor{UKDark}{HTML}{7a4f44}
\definecolor{UKDarker}{HTML}{4d312b} 
\definecolor{UKDarkest}{HTML}{2e1e1a}
\definecolor{UKRed}{HTML}{bf1f1c}

\setbeamertemplate{footline}[frame number]{}
\setbeamertemplate{navigation symbols}{}

%\usecolortheme{beaver}
\setbeamertemplate{itemize item}[square]
\setbeamercolor{itemize item}{fg = UKBrown}
\setbeamercolor{itemize subitem}{fg = UKLight}
\setbeamercolor{enumerate item}{fg = UKDark}

\setbeamercolor{footnote}{fg=UKLight}
\setbeamercolor{footnote mark}{fg=UKLight}
\setbeamerfont{footnote}{size=\tiny}
\renewcommand\footnoterule{}

\usetheme{default}
\beamertemplatenavigationsymbolsempty
\setbeamercolor{title}{fg=white, bg=UKBrown}
\setbeamercolor{frametitle}{fg=white, bg=UKBrown}
\setbeamercolor{block title}{bg=UKBrown, fg= white}
\setbeamercolor{block body}{bg =UKLight, fg = UKDarkest}

\setbeamercolor{block title alerted}{bg=UKOrange, fg= white}
\setbeamercolor{block body alerted}{bg =UKLight, fg = UKDarkest}


%\setbeamercolor{section in toc}{fg = UKBrown}
%\setbeamercolor{section in toc}{fg = UKDarkest}

% odstrani gulicky
\renewcommand*{\slideentry}[6]{}

\useoutertheme[subsection=false]{miniframes}
\AtBeginSection[]{\subsection{}}

\setbeamercolor{below lower separation line head}{bg=UKDark}
\addtobeamertemplate{headline}{}{%
  \begin{beamercolorbox}[colsep=0.5pt]{below lower separation line head}
  \end{beamercolorbox}
}
%\setbeamercolor*{mini frame}{fg=white,bg=UKRosy}
\setbeamercolor{section in head/foot}{fg=UKLight, bg=UKDark}

\usepackage{etoolbox}
\makeatletter
\preto{\@verbatim}{\topsep=0pt \partopsep=0pt }
\makeatother

%\setbeamertemplate{itemize/enumerate body begin}{\normalsize}
%\setbeamertemplate{itemize/enumerate subbody begin}{\normalsize}




%\newcommand{\codeblock}[2]{ \begin{block}{#1} \begin{verbatim}#2\end{verbatim}\end{block}}

%\defbeamertemplate*{title page}{customized}[1][]
%{
%  \begin{centering}
%    \begin{beamercolorbox}[sep=8pt,center]{title}
%      \usebeamerfont{title}\inserttitle
%    \end{beamercolorbox}
%  \end{centering}
%  \bigskip
%
%\begin{columns}[onlytextwidth,T]
%
%
%  \column{27mm}
%  \includegraphics[width=27mm]{images/logoFMFI.png}
%  
%  \column{\dimexpr\linewidth-54mm-6mm}
%  \centering
%  \vspace{5mm}  
%  \usebeamerfont{author}\insertauthor\par
%  \vspace{5mm}
%  \usebeamerfont{institute}\insertinstitute\par
%
%  \column{27mm}
%  \includegraphics[width=27mm]{images/logoUK.png}  
%\end{columns}
%\centering
%\vspace{7mm}
%  \usebeamerfont{date}\insertdate\par
%}

\DeclareMathOperator*{\argmin}{arg\,min}
\newcommand{\e}[1]{$\cdot 10^{#1}$}

%\newcommand{\codeblock}[2]{ \begin{block}{#1} \begin{verbatim}#2\end{verbatim}\end{block}}


\title[1. Úloha]{Pokročilé spracovanie obrazu - Úloha - GUI Hrany}
\author[Kocur]{Ing. Viktor Kocur \\{\small viktor.kocur@fmph.uniba.sk}}
\institute{DAI FMFI UK}
\date{30.10.2019}

\begin{document}
\selectlanguage{slovak}

\begin{frame}
  \titlepage
\end{frame}

\begin{frame}
\frametitle{Zadanie}
\begin{block}{GUI}
Cieľom úlohy je vytvoriť GUI, v ktorom bude možné načítať obrázok, zašumieť ho, vyhladiť ho a detekovať v ňom hrany.
\end{block}

\begin{block}{Špecifikácia}
Po načítaní obrázku bude zobrazený v axes. Všetky zmeny (zašumenie a hladenie) sa prejavia na zobrazenom obrázku. Výsledok po detekcii hrán sa zobrazí vedľa obrázka v samostatnom axes.
\end{block}


\begin{block}{Reset button}
Pridajte button, ktorý umožní zrušiť doterajšie úpravy načítaného obrázka (zobrazí ho znova, bez toho aby ho uživateľ musel hľadať v súboroch).
\end{block}
\end{frame}


\begin{frame}
\frametitle{Šum}
\begin{block}{Sol a korenie (Impulzný šum)}
Na pridanie šumu bude v GUI umiestnený button. Bude sa dať (napr. sliderom, textboxom) upraviť miera šumu pomocov dvoch prvkov jeden pre biely a čierny šum.
\end{block}

\begin{block}{Gaussovský aditívny šum}
Pre gaussovský šum bude taktiež button a bude sa dať určiť nejakým prvkom hodnota sigma.
\end{block}
\end{frame}

\begin{frame}
\frametitle{Vyhladzovanie a Hrany}

\begin{block}{Mediánová filtrácia}
Pridajte button na mediánovú filtráciu. 
\end{block}

\begin{block}{Gaussovské vyhladzovanie}
Pridajte button na gaussovské vyhladenie. Pridajte prvok, tak aby sa dala určiť sigma.
\end{block}

\begin{block}{Hrany}
Vytvorte gui prvok (buttongroup + radio button, alebo pop-up menu) ktorým bude možné vybrať si metódu detekcie hrán zo Sobel, Prewitt, Roberts a Canny. Pridajte button ktorým sa daná metóda použije a výsledok sa zobrazí vedľa načítaného obrázku.
\end{block}
\end{frame}

\begin{frame}
\frametitle{Odovzdávanie}
\begin{block}{Odovzdávanie}
Odovzdajte vami vytvorené .m a .fig súbory v zipe vo formáte priezvisko.zip. Odošlite ho na adresu kocurvik@gmail.com s predmetom PSO - DU1. Deadline je 17.11.2020 o 23:59. Úloha je na 7.5 boda. Za každý deň meškania -1.5 boda.
\end{block}

\begin{block}{Bodovanie}
Pri bodovaní sa bude prihliadať aj na rozloženie a logiku UI prvkov. Dajte si pozor na to, aby napríklad pri možnostiach volenia rôznych hodnôt bolo možné zvoliť také hodnoty ktoré sú pre danú veličinu relevantné a aby bolo jasné o aké hodnoty sa jedná v ktorom prípade. Namiesto funkcií ktoré sme si robili môžete použiť aj 'hotové' matlabovské funkcie, ale dajte si pozor aby ich parametre korešpondovali s našou definíciou.
\end{block}
\end{frame}
\end{document}