\documentclass{beamer}
%\usetheme{Ilmenau}
%\usecolortheme{beaver}

\usepackage[slovak,american]{babel}
\usepackage[utf8]{inputenc}
\usepackage{graphicx}
\usepackage{adjustbox}
\usepackage{xcolor}
\usepackage{mathrsfs}
 
 \newsavebox\MBox
\newcommand\Cline[2][red]{{\sbox\MBox{$#2$}%
  \rlap{\usebox\MBox}\color{#1}\rule[-2.2\dp\MBox]{\wd\MBox}{1pt}}}

%\usefonttheme{serif}

%\definecolor{UKOrange}{HTML}{ef9424} %
\definecolor{UKOrange}{HTML}{7a2c18} %
\definecolor{UKBrown}{HTML}{a96d5e} %
\definecolor{UKLight}{HTML}{d8b6ab} %
\definecolor{UKDark}{HTML}{7a4f44}
\definecolor{UKDarker}{HTML}{4d312b} 
\definecolor{UKDarkest}{HTML}{2e1e1a}
\definecolor{UKRed}{HTML}{bf1f1c}

\setbeamertemplate{footline}[frame number]{}
\setbeamertemplate{navigation symbols}{}

%\usecolortheme{beaver}
\setbeamertemplate{itemize item}[square]
\setbeamercolor{itemize item}{fg = UKBrown}
\setbeamercolor{itemize subitem}{fg = UKLight}
\setbeamercolor{enumerate item}{fg = UKDark}

\setbeamercolor{footnote}{fg=UKLight}
\setbeamercolor{footnote mark}{fg=UKLight}
\setbeamerfont{footnote}{size=\tiny}
\renewcommand\footnoterule{}

\usetheme{default}
\beamertemplatenavigationsymbolsempty
\setbeamercolor{title}{fg=white, bg=UKBrown}
\setbeamercolor{frametitle}{fg=white, bg=UKBrown}
\setbeamercolor{block title}{bg=UKBrown, fg= white}
\setbeamercolor{block body}{bg =UKLight, fg = UKDarkest}

\setbeamercolor{block title alerted}{bg=UKOrange, fg= white}
\setbeamercolor{block body alerted}{bg =UKLight, fg = UKDarkest}


%\setbeamercolor{section in toc}{fg = UKBrown}
%\setbeamercolor{section in toc}{fg = UKDarkest}

% odstrani gulicky
\renewcommand*{\slideentry}[6]{}

\useoutertheme[subsection=false]{miniframes}
\AtBeginSection[]{\subsection{}}

\setbeamercolor{below lower separation line head}{bg=UKDark}
\addtobeamertemplate{headline}{}{%
  \begin{beamercolorbox}[colsep=0.5pt]{below lower separation line head}
  \end{beamercolorbox}
}
%\setbeamercolor*{mini frame}{fg=white,bg=UKRosy}
\setbeamercolor{section in head/foot}{fg=UKLight, bg=UKDark}

\usepackage{etoolbox}
\makeatletter
\preto{\@verbatim}{\topsep=0pt \partopsep=0pt }
\makeatother

%\setbeamertemplate{itemize/enumerate body begin}{\normalsize}
%\setbeamertemplate{itemize/enumerate subbody begin}{\normalsize}




%\newcommand{\codeblock}[2]{ \begin{block}{#1} \begin{verbatim}#2\end{verbatim}\end{block}}

%\defbeamertemplate*{title page}{customized}[1][]
%{
%  \begin{centering}
%    \begin{beamercolorbox}[sep=8pt,center]{title}
%      \usebeamerfont{title}\inserttitle
%    \end{beamercolorbox}
%  \end{centering}
%  \bigskip
%
%\begin{columns}[onlytextwidth,T]
%
%
%  \column{27mm}
%  \includegraphics[width=27mm]{images/logoFMFI.png}
%  
%  \column{\dimexpr\linewidth-54mm-6mm}
%  \centering
%  \vspace{5mm}  
%  \usebeamerfont{author}\insertauthor\par
%  \vspace{5mm}
%  \usebeamerfont{institute}\insertinstitute\par
%
%  \column{27mm}
%  \includegraphics[width=27mm]{images/logoUK.png}  
%\end{columns}
%\centering
%\vspace{7mm}
%  \usebeamerfont{date}\insertdate\par
%}

\DeclareMathOperator*{\argmin}{arg\,min}
\newcommand{\e}[1]{$\cdot 10^{#1}$}

%\newcommand{\codeblock}[2]{ \begin{block}{#1} \begin{verbatim}#2\end{verbatim}\end{block}}


\title[9. cvičenie]{Pokročilé spracovanie obrazu - Fourierová transformácia}
\author[Kocur]{Ing. Viktor Kocur \\{\small viktor.kocur@fmph.uniba.sk}}
\institute{DAI FMFI UK}
\date{20.11.2019}

\begin{document}
\selectlanguage{slovak}


\begin{frame}

  \titlepage

\end{frame}

\section{Fourierová transformácia}

\subsection{Matematické základy}

\begin{frame}
\frametitle{Diskrétna Fourierová transformácia - 1D} 
  \begin{block}{Definícia}    
    $$F_k = \mathscr{F}[ \vec{f}]_k = \sum_{n = 0}^{N - 1} f_n \cdot e^{\frac{-2 \pi i n k}{N}}$$
  \end{block}

  \begin{block}{Inverzne}
    $$f_n = \mathscr{F}^{-1}[\vec{F}]_n = \frac{1}{N} \sum_{k = 0}^{N - 1} F_k \cdot e^{\frac{2 \pi i n k}{N}}$$
  \end{block}

\end{frame}

\begin{frame}
\frametitle{2D FFT v matlabe} 

\begin{block}{fft2}
fft2(I) - Vráti Fourierovú transformáciu šedotónového obrazu I.
\end{block}

\begin{block}{ifft2}
ifft2(F) - vráti inverznú Fourierovú transformáciu obrazu F.
\end{block}

\begin{block}{fftshift}
fftshift(fft2(I)) - Vráti Fourierovú transformáciu obrazu šedotónového obrazu I, tak že nulová frekvencia bude v strede obrazu.
\end{block}

\begin{block}{Úloha}
Skúste si zobraziť Fourierovú transformáciu zátišia.
\end{block}

\end{frame}

\begin{frame}
\frametitle{Komplexné čísla}
  
  \begin{alertblock}{Problém I - Komplexné čísla}
    FFT v matlabe vracia komplexné čísla! IFFT vracia komplexné čísla tiež. Musíme sa preto naučiť pracovať s komplexnými číslami.
  \end{alertblock}
  
  \begin{block}{Práca s komplexnými číslami}
  \begin{itemize}
  \item i, j - konštanty reprezentujúce komplexné číslo - nedá sa použiť ak máme definované premenné s rovnakým menom
  \item real(c) - reálna časť komlexného čisla c
  \item imag(c) - imaginárna časť komlexného čísla c
  \item abs(c) - absolútna hodnota komlexného čísla c
  \item angle(c) - uhol komplexného čísla c
  \end{itemize}
  \end{block}
\end{frame}

\begin{frame}[fragile]
\frametitle{Úloha} 

  \begin{alertblock}{Problém II}
    Potreba pracovať v double, komplexné čísla, fftshift, hodnoty transformácie sú príliš veľké v (0,0)!
  \end{alertblock}

  \begin{block}{Funkcia ako riešenie}
  \begin{verbatim}
function F = zobrfft(I)
    F = fftshift(fft2(im2double(I)));
    imagesc(log(abs(F)+1));
    colormap(gray);
end \end{verbatim}
  \end{block}    
    
\end{frame}

\begin{frame}[fragile]
\frametitle{Úloha} 

  \begin{block}{Úloha}
    Zobrazte si absolútne hodnoty fourierovej transformácie obrázkov pruhyhoriz.pgm, pruhyvert.pgm, pruhysikme.pgm, zatisie.pgm, boat.pgm, waveboat.pgm. Pre niektoré obrázky zobrazte fázy fourierovej transformácie.
  \end{block}
  
    \begin{block}{Úloha}
    Obrázok boat.pgm transformujte a potom použite inverznú transformáciu, ale využite len informáciu o fáze, alebo len informáciu o absolútnej hodnote, len reálnu časť, alebo len imaginárnu. Skúste v transformovanom obraze zachovať iba jeden z kvadrantov a takýto obraz inverzne transformovať.
  \end{block}  
\end{frame}

\subsection{Filtrácia v spektrálnej oblasti}

\begin{frame}
\frametitle{Filtrácia v spektrálnej oblasti - princíp}
  \begin{block}{Rozloženie frekvencií v matici po transformácii}
    V spektrálnej oblasti (po transformácii) sú nižšie frekvencie bližsie pri strede a vyššie na okrajoch.
  \end{block}
  
  \begin{block}{Význam}
    Vysoké frekvencie predstavujú detaily. Ak zachováme v obrázku len vysokospektrálne príspevky, tak dostaneme najmä hrany. Ak zachováme len príspevky nízkych frekvencií dostaneme rozmazaný obrázok.
  \end{block}
\end{frame}

\begin{frame}
\frametitle{Ideálny highpass a lowpass}
  \begin{block}{Highpass}
    Transformujeme obrázok. Hodnoty pre nízke frekvencie vynulujeme. Urobíme inverznú transformáciu.
  \end{block}
  
  \begin{block}{Lowpass}
    Rovnaký proces ale vynulujeme vysoké frekvencie.
  \end{block}
  
  \begin{block}{Cut-off frekvencia}
    Frekvencia od (do) ktorej nulujeme hodnoty sa nazýva frekvencia.
  \end{block}
\end{frame}

\begin{frame}[fragile]
\frametitle{Butterworthov filter}
  \begin{block}{Butterworthov filter}
  \begin{equation*}
    H = \frac{1}{1+\left(\sqrt{2} - 1\right) \left(\frac{D}{D_0}\right)^{2n}}
  \end{equation*}
  D je Euklidovská vzdialenosť od stredu.
  \end{block}

  \begin{block}{Funkcia na highpass Butterworthov filter}
    Otvorte si butterhigh.m zo zipu.
  \end{block}
  
  \begin{block}{Úloha}
    Zobrazte si pomocou imagesc tento filter. Použite tento filter na nejaký obrázok v spektrálnej oblasti. Upravte funkciu aby sme ju mohli použiť na lowpass filter.
  \end{block}    
\end{frame}

\begin{frame}
\frametitle{Ideálny highpass a lowpass}
  \begin{block}{Úloha}
  Vytvorte funkcie ktoré vrátia ideálny highpass a ideálny lowpass filter. Použite časť kódu z butterhigh.m.
  \end{block}
  
  \begin{block}{Úloha}
  Aplikujte filtre s rôznymi cut-off frekvenciami na boat.pgm a zatisie.pgm.
  \end{block}
\end{frame}

\begin{frame}
\frametitle{Filtrácia periodického šumu}
  \begin{block}{Periodický šum}
  Periodický šum sa vo Fourierovom obraze prejavý ako pár, alebo viac oblastí kde je zvýšená amplitúda. Vďaka tomu môžeme Fourierov obraz upraviť a previesť inverznú transformáciu. Vo výslednom obraze bude periodický šum potlačený.
  \end{block}
  
  \begin{block}{Úloha}
  Pre obrázok tree.png, waveboat.pgm a periodic.png potlačte pomcou Fourierovej transformácie periodický šum.
  \end{block}
\end{frame}



\end{document}