\documentclass{beamer}
%\usetheme{Ilmenau}
%\usecolortheme{beaver}

\usepackage[slovak,american]{babel}
\usepackage[utf8]{inputenc}
\usepackage{graphicx}
\usepackage{adjustbox}
\usepackage{xcolor}
\usepackage{mathrsfs}
 
 \newsavebox\MBox
\newcommand\Cline[2][red]{{\sbox\MBox{$#2$}%
  \rlap{\usebox\MBox}\color{#1}\rule[-2.2\dp\MBox]{\wd\MBox}{1pt}}}

%\usefonttheme{serif}

%\definecolor{UKOrange}{HTML}{ef9424} %
\definecolor{UKOrange}{HTML}{7a2c18} %
\definecolor{UKBrown}{HTML}{a96d5e} %
\definecolor{UKLight}{HTML}{d8b6ab} %
\definecolor{UKDark}{HTML}{7a4f44}
\definecolor{UKDarker}{HTML}{4d312b} 
\definecolor{UKDarkest}{HTML}{2e1e1a}
\definecolor{UKRed}{HTML}{bf1f1c}

\setbeamertemplate{footline}[frame number]{}
\setbeamertemplate{navigation symbols}{}

%\usecolortheme{beaver}
\setbeamertemplate{itemize item}[square]
\setbeamercolor{itemize item}{fg = UKBrown}
\setbeamercolor{itemize subitem}{fg = UKLight}
\setbeamercolor{enumerate item}{fg = UKDark}

\setbeamercolor{footnote}{fg=UKLight}
\setbeamercolor{footnote mark}{fg=UKLight}
\setbeamerfont{footnote}{size=\tiny}
\renewcommand\footnoterule{}

\usetheme{default}
\beamertemplatenavigationsymbolsempty
\setbeamercolor{title}{fg=white, bg=UKBrown}
\setbeamercolor{frametitle}{fg=white, bg=UKBrown}
\setbeamercolor{block title}{bg=UKBrown, fg= white}
\setbeamercolor{block body}{bg =UKLight, fg = UKDarkest}

\setbeamercolor{block title alerted}{bg=UKOrange, fg= white}
\setbeamercolor{block body alerted}{bg =UKLight, fg = UKDarkest}


%\setbeamercolor{section in toc}{fg = UKBrown}
%\setbeamercolor{section in toc}{fg = UKDarkest}

% odstrani gulicky
\renewcommand*{\slideentry}[6]{}

\useoutertheme[subsection=false]{miniframes}
\AtBeginSection[]{\subsection{}}

\setbeamercolor{below lower separation line head}{bg=UKDark}
\addtobeamertemplate{headline}{}{%
  \begin{beamercolorbox}[colsep=0.5pt]{below lower separation line head}
  \end{beamercolorbox}
}
%\setbeamercolor*{mini frame}{fg=white,bg=UKRosy}
\setbeamercolor{section in head/foot}{fg=UKLight, bg=UKDark}

\usepackage{etoolbox}
\makeatletter
\preto{\@verbatim}{\topsep=0pt \partopsep=0pt }
\makeatother

%\setbeamertemplate{itemize/enumerate body begin}{\normalsize}
%\setbeamertemplate{itemize/enumerate subbody begin}{\normalsize}




%\newcommand{\codeblock}[2]{ \begin{block}{#1} \begin{verbatim}#2\end{verbatim}\end{block}}

%\defbeamertemplate*{title page}{customized}[1][]
%{
%  \begin{centering}
%    \begin{beamercolorbox}[sep=8pt,center]{title}
%      \usebeamerfont{title}\inserttitle
%    \end{beamercolorbox}
%  \end{centering}
%  \bigskip
%
%\begin{columns}[onlytextwidth,T]
%
%
%  \column{27mm}
%  \includegraphics[width=27mm]{images/logoFMFI.png}
%  
%  \column{\dimexpr\linewidth-54mm-6mm}
%  \centering
%  \vspace{5mm}  
%  \usebeamerfont{author}\insertauthor\par
%  \vspace{5mm}
%  \usebeamerfont{institute}\insertinstitute\par
%
%  \column{27mm}
%  \includegraphics[width=27mm]{images/logoUK.png}  
%\end{columns}
%\centering
%\vspace{7mm}
%  \usebeamerfont{date}\insertdate\par
%}

\DeclareMathOperator*{\argmin}{arg\,min}
\newcommand{\e}[1]{$\cdot 10^{#1}$}

%\newcommand{\codeblock}[2]{ \begin{block}{#1} \begin{verbatim}#2\end{verbatim}\end{block}}




\title[2. Úloha]{Pokročilé spracovanie obrazu - Úloha - Segmentácia}
\author[Kocur]{Ing. Viktor Kocur \\{\small viktor.kocur@fmph.uniba.sk}}
\institute{DAI FMFI UK}
\date{3.12.2020}

\begin{document}
\selectlanguage{slovak}

\begin{frame}
  \titlepage
\end{frame}

\begin{frame}
\frametitle{Zadanie}
\begin{block}{Cieľ}
Cieľom úlohy je otestovať si segmentačné metódy a popísať postupy ktoré ste použili v pdf súbore.
\end{block}

\begin{block}{Výstup}
Výstupom vašej práce bude pdf súbor s popisom vášho postupu a výsledkami a kód ktorý ste použili na dosiahnutie týchto výsledkov.
\end{block}
\end{frame}


\begin{frame}
\frametitle{Dáta}
\begin{block}{Dataset}
\href{https://www2.eecs.berkeley.edu/Research/Projects/CS/vision/bsds/BSDS300/html/dataset/images.html}{The Berkeley Segmentation Dataset and Benchmark}
\end{block}

\begin{block}{Obrázky}
Vyberte si 3 obrázky z tohoto datasetu. K datasetu je aj ground truth. Pre jeden obrázok si ju stiahnite a vďaka nej si spravte ručne (v matlab image segmenter appke) segmentáciu nejakého objektu.
\end{block}
\end{frame}

\begin{frame}
\frametitle{Segmentácia}

\begin{block}{Tri metódy}
Pre každý obrázok použite k-means, graph cut a segmentáciu na základe farby na vytvorenie segmentovaného obrázku, buď úplného, alebo len nejakého objektu.  Pohrajte sa s parametrami. Popíšte v pdfku ako ste postupovali a pridajte doň aj obrázky s (medzi) výsledkami. Pre každý obrázok musíte použiť všetky tri metódy.
\end{block}

\end{frame}

\begin{frame}
\frametitle{Segmentácia}

\begin{block}{Porovnanie}
Na porovananie výsledkov použijete ručne segmentovaný obrázok a segmentáciu podľa datasetu. Pre každú z metód vypočítajte tzv. IoU metriku. Tá sa počíta ako podiel medzi obsahom prieniku segmentovanej oblasti a ground truth segmentácii a obsahom ich zjednotenia.
\end{block}

\begin{alertblock}{Pozn}
IoU metriku počítajte po pixeloch. Najjednoduchšie je využiť logické operácie.
\end{alertblock}
\end{frame}

\begin{frame}
\frametitle{Odovzdávanie}
\begin{block}{Odovzdávanie}
Odošlite pdf súbor, vybraté obrázky, ručnú segmentáciu a vami použité skripty pre k-means a segmentáciu na základe farby ako aj výpočet IoU metriky na kocurvik@gmail.com Skripty skúste spraviť, tak aby sa zakomentovaním/odkomentovaním dali skontrolovať postupy pre rôzne obrázky.
\end{block}

\begin{block}{Bodovanie}
Pri bodovaní sa bude prihliadať na dostatočný popis postupu, výsledky a správnosť odoslaného kódu. Maximálne sa dá získať 7.5 boda. Deadline 12.12. 23:59 je  Za každý deň meškania je -1.5 boda.
\end{block}
\end{frame}
\end{document}