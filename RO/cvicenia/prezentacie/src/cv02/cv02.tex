\documentclass{beamer}
%\usetheme{Ilmenau}
%\usecolortheme{beaver}

\usepackage[slovak,american]{babel}
\usepackage[utf8]{inputenc}
\usepackage{graphicx}
\usepackage{adjustbox}
 \usepackage{xcolor}
 
 \newsavebox\MBox
\newcommand\Cline[2][red]{{\sbox\MBox{$#2$}%
  \rlap{\usebox\MBox}\color{#1}\rule[-2.2\dp\MBox]{\wd\MBox}{1pt}}}

%\usefonttheme{serif}

\definecolor{UKOrange}{HTML}{ef9424} %
\definecolor{UKBrown}{HTML}{a96d5e} %
\definecolor{UKLight}{HTML}{d8b6ab} %
\definecolor{UKDark}{HTML}{7a4f44}
\definecolor{UKDarker}{HTML}{4d312b} 
\definecolor{UKDarkest}{HTML}{2e1e1a}
\definecolor{UKRed}{HTML}{bf1f1c}

\setbeamertemplate{footline}[frame number]{}
\setbeamertemplate{navigation symbols}{}

%\usecolortheme{beaver}
\setbeamertemplate{itemize item}[square]
\setbeamercolor{itemize item}{fg = UKBrown}
\setbeamercolor{itemize subitem}{fg = UKLight}
\setbeamercolor{enumerate item}{fg = UKDark}

\setbeamercolor{footnote}{fg=UKLight}
\setbeamercolor{footnote mark}{fg=UKLight}
\setbeamerfont{footnote}{size=\tiny}
\renewcommand\footnoterule{}

\usetheme{default}
\beamertemplatenavigationsymbolsempty
\setbeamercolor{title}{fg=white, bg=UKBrown}
\setbeamercolor{frametitle}{fg=white, bg=UKBrown}
\setbeamercolor{block title}{bg=UKBrown, fg= white}
\setbeamercolor{block body}{bg =UKLight, fg = UKDarkest}

\useoutertheme[subsection=false]{miniframes}
\AtBeginSection[]{\subsection{}}

\setbeamercolor{below lower separation line head}{bg=UKDark}
\addtobeamertemplate{headline}{}{%
  \begin{beamercolorbox}[colsep=0.5pt]{below lower separation line head}
  \end{beamercolorbox}
}
%\setbeamercolor*{mini frame}{fg=white,bg=UKRosy}
\setbeamercolor{section in head/foot}{fg=UKLight, bg=UKDark}

%\setbeamertemplate{itemize/enumerate body begin}{\normalsize}
%\setbeamertemplate{itemize/enumerate subbody begin}{\normalsize}




%\newcommand{\codeblock}[2]{ \begin{block}{#1} \begin{verbatim}#2\end{verbatim}\end{block}}

%\defbeamertemplate*{title page}{customized}[1][]
%{
%  \begin{centering}
%    \begin{beamercolorbox}[sep=8pt,center]{title}
%      \usebeamerfont{title}\inserttitle
%    \end{beamercolorbox}
%  \end{centering}
%  \bigskip
%
%\begin{columns}[onlytextwidth,T]
%
%
%  \column{27mm}
%  \includegraphics[width=27mm]{images/logoFMFI.png}
%  
%  \column{\dimexpr\linewidth-54mm-6mm}
%  \centering
%  \vspace{5mm}  
%  \usebeamerfont{author}\insertauthor\par
%  \vspace{5mm}
%  \usebeamerfont{institute}\insertinstitute\par
%
%  \column{27mm}
%  \includegraphics[width=27mm]{images/logoUK.png}  
%\end{columns}
%\centering
%\vspace{7mm}
%  \usebeamerfont{date}\insertdate\par
%}


\title[Štatistika]{Rozpoznávanie obrazcov - 2. cvicečenie \\ Štatistika}
\author[Viktor Kocur]{Viktor Kocur \\{\small viktor.kocur@fmph.uniba.sk}}
\institute{DAI FMFI UK}
\date{26.2.2019}
%\titlegraphic{\includegraphics[width=2.7cm]{images/logoFMFI.png}\hspace*{1cm}~%
%   \includegraphics[width=2.7cm]{images/logoUK.png}
%}


\begin{document}
\selectlanguage{slovak}

\begin{frame}[plain]
  \titlepage  
\end{frame}

\section{Definície}
%\subsection{Introduction}

\begin{frame}
\frametitle{Klasické definície}

\begin{block}{Klasická}
Pravdepodobnosť javu $A$ je jeho relatívna početnosť v pokusoch:
\begin{equation*}
P(A) = \frac{N_A}{N}
\end{equation*}
\end{block}

\begin{block}{Limitná}
Pravdepodobnosť javu $A$ je jeho relatívna početnosť v nekonečne pokusov:
\begin{equation*}
P(A) = \lim_{n\to\infty} \frac{N_A}{N}
\end{equation*}
\end{block}
\end{frame}

\begin{frame}
\frametitle{Axiomatická definícia - $sigma$-algebra}

\begin{block}{$\sigma$-algebra}
Pole javov $\mathcal{A}$ je $\sigma$-algebra na $\Omega$ teda preň platia:
\begin{gather}
\Omega \in \mathcal{A} \\
A \in \mathcal{A} \implies A^{c} \in \mathcal{A} \\
(\forall n \in \mathbb{N}) (A_n \in \mathcal{A}) \implies \bigcup_{n \in \mathbb{N}} A_n \in \mathcal{A}
\end{gather}
\end{block}

\begin{block}{Elementárny jav}
Elementárny jav je taký jav z $\Omega$, ktorý sa nedá rozložiť na iné javy.
\end{block}
\end{frame}

\begin{frame}
\frametitle{Axiomatická definícia - pravdepodobnosť}

\begin{block}{Pravdepodobnosť}
Pravdepodobnosť je funkcia $P: \mathcal{A} \rightarrow \left<0,1\right>$, taká že spĺňa:
\begin{gather}
P(\Omega) = 1 \\
P\left(\bigcup_{n \in \mathbb{N}} A_n\right) = \sum_{n \in \mathbb{N}} P(A_n),
\end{gather}
\centering
ak $A_n$ je postupnosť po dvoch disjunktných javov.
\end{block}

\begin{block}{Pravdepodobnostný priestor}
Pravdepodobnostný priestor je trojica $\left(\Omega, \mathcal{A}, P \right)$.
\end{block}
\end{frame}


\begin{frame}
\frametitle{Operácie}

\begin{block}{Zjednotenie}
Zjednotenie $A \cup B$ nastane ak nastane aspoň jeden z javov.
\end{block}

\begin{block}{Prienik}
Prienik $A \cap B$ nastane ak nastanú oba javy.
\end{block}

\begin{block}{Opačný jav}
Opačný jav k $A$ je $A^c$.
\end{block}

\begin{block}{Nezlučiteľné javy}
Ak $A \cap B = \emptyset$ tak sú to nezlučiteľné javy.
\end{block}
\end{frame}

\section{Príklady}

\begin{frame}
\frametitle{Príklad 1}
Aká je pravdepodobnosť, že pri hode kockou:
\begin{itemize}
\item Padne číslo X \onslide<3->{$= \frac{1}{6}$}
\begin{itemize}
\item počet priaznivých elementárnych javov  \onslide<2->{= 1}
\item celkový počet elementárnych javov  \onslide<2->{= 6}
\end{itemize}
\item Padne nepárne číslo \onslide<5->{$= \frac{1}{2}$}
\begin{itemize}
\item počet priaznivých elementárnych javov \onslide<4->{= 3}
\item celkový počet elementárnych javov \onslide<4->{= 6}
\end{itemize}
\end{itemize}
\end{frame}

\begin{frame}
\frametitle{Príklad 2}
Dodávka obsahuje 50 matíc a 150 skrutiek. Polovica matíc a polovica skrutiek je hrdzavá. Ak náhodne vyberieme jednu súčiastku, aká je pravdepodobnosť, že to bude matica alebo, že to bude hrdzavá súčiastka?

\onslide<2-> Pokus má 200 možných výsledkov.  \\
\begin{itemize}
\item<3-> Označme $A$ súčiastka je hrdzavá - 100.
\item<3-> Označme $B$ súčiastka je matica - 50.
\item<4-> Označme $A \cap B$ súčiastka je hrdzavá matica - 25.
\item<5-> $P(A \cup B) = P(A) + P(B) - P(A \cap B) = \frac{100}{200} + \frac{50}{200} - \frac{25}{200} = \frac{5}{8}$
\end{itemize}
\end{frame}


\begin{frame}
\frametitle{Podmienená pravdepodobnosť}

\begin{block}{Podmienená pravdepodobnosť}
Pravdepodobnosť že nastane jav $A$ za podmienky, že nastal jav $B$:
\begin{equation*}
 P(A|B) = \frac{P(A \cap B)}{P(B)}
\end{equation*}
\end{block}
\end{frame}

\begin{frame}
\frametitle{Príklad 3}
V rodine sú dve deti. Určite pravdepodobnosť, že obe deti sú chlapci, ak vieme, že jedno z detí je chlapec.
\begin{itemize}
\item<2-> $\Omega = \{ (c,c), (c,d), (d,c), (d,d) \} $ \\
\item<3-> Označme $A$ obe deti sú chlapci $\{(c,c)\}$.
\item<3-> Označme $B$ aspoň jedno dieťa je chlapec  $\{ (c,c), (c,d), (d,c) \} $ 
\item<4-> $P(A | B) = \frac{P(A \cap B)}{P(B)} = \frac{\frac{1}{4}}{\frac{3}{4}} = \frac{1}{3}$
\end{itemize}
\end{frame}

\begin{frame}
\frametitle{Veta o úplnej pravdepodobnosti}

\begin{block}{Rozklad výberového priestoru}
Množiná navzájom disjunktných javov $\{B_1, ..., B_n\}$ z $\Omega$ tvorí rozklad výberového priestoru ak $\cup_{i \in \hat{n}} B_i = \Omega$.
\end{block}

\begin{block}{Veta o úplnej pravdepodobnosti}
Nech  $\{B_1, ..., B_n\}$ tvorí rozklad výberového priestoru $\Omega$. Porom pre jav $A \in \Omega$ platí:
\begin{equation*}
P(A) = \sum_{k \in \hat{n}} P(A|B_i)P(B_i)
\end{equation*}
\end{block}
\end{frame}

\begin{frame}
\frametitle{Príklad 4}
Elektrické žiarovky sú vyrábané 3 závodmi. Prvý dodáva 25\%, druhý 40\% a tretí 35\% celej dodávky. Z celkovej produkcie 1. závodu je 88\%, 2. závodu 75\%, 3. závodu 85\% štandardných. Aká je pravdepodobnosť, že náhodne vybraná žiarovka je štandardná?

\begin{itemize}
\item<2-> $B_i$ je pravdepodonosť že žiarovka bola vyrobená v i-tom závode \\
\item<3-> $P(B_1) = 0.25, P(B_2) = 0.4, P(B_3) = 0.35$
\item<4-> $P(A | B_1) = 0.88, P(A | B_2) = 0.75, P(A | B_3) = 0.85$
\item<5-> $P(A) = \sum_{k \in \hat{3}} P(A|B_i)P(B_i)$
\item<6-> $P(A) = 0.88 \cdot 0.25 + 0.75 \cdot 0.4 + 0.85 \cdot 0.35 = 0.8175$ 
\end{itemize}
\end{frame}

\begin{frame}
\frametitle{Príklad 5}
Elektrónka zapojená v televízore môže byť od 4 rôznych výrobcov s pravdepodobnosťami 0,2; 0,3; 0,35; 0,15 pre jednotlivých výrobcov. Pravdepodobnosti, že elektrónky od jednotlivých výrobcov vydržia predpísaný počet hodín sú postupne 0,45; 0,60; 0,75; 0,30. Aká je pravdepodobnosť, že náhodne vybratá elektrónka vydrží predpísaný počet hodín?

\begin{itemize}
\item<2-> $B_i$ je pravdepodonosť že žiarovka bola vyrobená v i-tym výrobcom \\
\item<3-> $P(B_1) = 0.2, P(B_2) = 0.3, P(B_3) = 0.35, P(B_4) = 0.15$
\item<4-> $P(A | B_1) = 0.45, P(A | B_2) = 0.6, P(A | B_3) = 0.75, P(A | B_4) = 0.3$
\item<5-> $P(A) = \sum_{k \in \hat{4}} P(A|B_i)P(B_i)$
\item<6-> $P(A) = 0.45 \cdot 0.2 + 0.6 \cdot 0.3 + 0.75 \cdot 0.35 +  0.3 \cdot 0.15= 0.5775$ 
\end{itemize}
\end{frame}


\begin{frame}
\frametitle{Príklad 6}
V terénnej súťaži spoľahlivosti zostalo 10 motocyklov prvej série, ktoré vykazovali spoľahlivosť jazdy 85\%, 8 motocyklov druhej série, ktoré vykazovali spoľahli-vosť 75\%, 5 motocyklov tretej série, ktoré vykazovali spoľahlivosť 60\%. Aká je pravdepodobnosť, že nasledujúci deň náhodne vybratý motocykel nedošiel?

\begin{itemize}
\item<2-> Označíme ako $A$ pravdepodobnosť že motorka dojde.\\
\item<3-> $B_i$ je spoľahlivosť motorky i-tej série.
\item<4-> $P(A) = \sum_{k \in \hat{3}} P(A|B_i)P(B_i)$
\item<5-> $P(A) = 0.85 \cdot \frac{8}{23} + 0.75 \cdot \frac{8}{23} + 0.6 \cdot \frac{5}{23} = 0.24$ 
\item<6-> Výsledok = $P(A^c) = 1 - P(A) = 0.76$
\end{itemize}
\end{frame}



\begin{frame}
\frametitle{Príklad 7}
V pivnici je tma. Kompóty sú poukladané na policiach. Na prvej polici je 6 čučoriedkových a 10 malinových kompótov, na druhej polici je 7 čučoriedkových a 5 malinových kompótov. Marienka preložila jeden kompót z prvej police na druhú. Janko si potom vybral jeden kompót z a) prvej police, b) druhej police. Aká je pravdepodobnosť, že si vybral malinový?

Riešenie a)
\begin{itemize}
\item<2-> $B_1$ - 5 čučoriedkových a 10 malinových na prvej polici\\
\item<2-> $B_2$ - 6 čučoriedkových a 9 malinových na prvej polici\\
\item<3-> $P(A|B_1) = \frac{10}{15}, P(A|B_2) = \frac{9}{15}$
\item<4-> $P(A) = \frac{10}{15} \cdot \frac{6}{16} +  \frac{9}{15} \cdot \frac{10}{16} = 0.625$ 
\end{itemize}
\end{frame}

\begin{frame}
\frametitle{Príklad 8}
Katarína A., Lenka B. a Monika C. sa spolu učia na maturitu. Z 30 maturitných otázok z angličtiny sa však stihli naučiť len prvých 20. Aká je pravdepodobnosť, že a) Lenka b) Monika si vytiahne otázku, ktorú vie, ak odpovedajú v poradí podľa abecedy a každá musí mať inú otázku?

Riešenie a)
\begin{itemize}
\item<2-> $B_1$ - katka vytiahne otázku, ktorú vie\\
\item<2-> $B_2$ -  katka vytiahne otázku, ktorú nevie\\
\item<3-> $P(A|B_1) = \frac{20}{30}, P = \frac{10}{30}$
\item<4-> $P(A) = \frac{19}{29} \cdot \frac{20}{30} +  \frac{20}{29} \cdot \frac{10}{30} = \frac{2}{3}$ 
\end{itemize}
\end{frame}

\begin{frame}
\frametitle{Bayesov vzorec}

\begin{block}{Bayesov vzorec}
\begin{equation*}
P(A|B) = \frac{P(B|A)P(A)}{P(B)}
\end{equation*}
\end{block}
\end{frame}

\begin{frame}
\frametitle{Príklad 8}
Karolovi urobili v rámci prehliadky Rtg hrudníku, ktorý sa vrátil s pozitívnym nálezom na rakovinu pľúc. Aká je pravdepodobnosť, že Karol má rakovinu, ak vieme, že test je falošne pozitívny v 5\% prípadov a falošne negatívny v 40\% prípadov. A tiež vieme, že iba jeden z 500 zamestnancov v Karolovej pozícii má rakovinu.

\begin{itemize}
\item<2-> $P(A_p) = 0.002$ - chorý, $P(A_n) = 0.998$ - zdravý
\item<2-> $B$ - pozitívny test
\item<3-> $P(B| A_p) = 0.6, P(B|A_n) = 0.05$
\item<4-> $P(A_p | B) = \frac{0.6 \cdot 0.002}{0.6 \cdot 0.002 + 0.05 \cdot  0.998} = 0.024$ 
\end{itemize}
\end{frame}

\begin{frame}
\frametitle{Príklad 9}
Rovnaký test ako Karol podstúpil aj Peter a tiež mu vyšiel pozitívny test. Peter však 20 rokov pracoval ako baník v banskom dole. Peter vie, že 15\% jeho bývalých kolegov má rakovinu pľúc. Ak sa ostatné podmienky nezmenili, aká je pravdepo-dobnosť, že rakovinu má aj Peter?

\begin{itemize}
\item<2-> $P(A_p) = 0.15$ - chorý, $P(A_n) = 0.85$ - zdravý
\item<2-> $B$ - pozitívny test
\item<3-> $P(B| A_p) = 0.6, P(B|A_n) = 0.05$
\item<4-> $P(A_p) = \frac{0.6 \cdot 0.15}{0.6 \cdot 0.15 + 0.05 \cdot 0.85} = 0.679$ 
\end{itemize}
\end{frame}

\begin{frame}
\frametitle{Príklad 10}
Predpokladajme, že máme školu so 60\% chlapcami a 40\% dievčatami. Všetci chlapci nosia nohavice a z dievčat nosí nohavice polovica. Pozorovateľ vidí z diaľky študenta v nohaviciach. Aká je pravdepodobnosť, že ten náhodný študent je dievča?

\begin{itemize}
\item<2-> $P(A_c) = 0.6$ - chlapec, $P(A_d) = 0.4$ - dievča
\item<2-> $B$ - osoba má nohavice
\item<3-> $P(B| A_c) = 1, P(B|A_d) = 0.5$
\item<4-> $P(A_p) = \frac{0.5 \cdot 0.4}{0.5 \cdot 0.4 + 0.6} = 0.25$ 
\end{itemize}
\end{frame}

\begin{frame}
\frametitle{Náhodná premenná}

\begin{block}{Náhodná premenná}
Funkcia, ktorej hodnota je určená výsledkom náhodného pokusu. Priraďuje číselnú hodnotu každému javu.
\end{block}

\begin{block}{Distribučná funkcia}
Opisuje rozdelenie pravdepodobnosti náhodnej premennej definovanej na pravdepodobnostnom priestore.
\end{block}
\end{frame}

\begin{frame}
\frametitle{Príklad 11}
Máme 4 stroje (nezávislé na sebe): 1. sa pokazí s pravdepodobnosťou 10\%, 2. s pravdepodobnosťou 15\%, 3. s pravdepodobnosťou 30\%, 4. stroj spravdepodobnosťou 50\%. Náhodná premenná X označuje počet pokazených strojov. Popíšte túto náhodnú premennú. T.j. popíšte s akými pravdepodobnosťami bude mať jednotlivé hodtnoty $\{0, ...,4\}$.

\begin{itemize}
\item<2-> $P(X = 4) = 0.1 \cdot 0.15  \cdot 0.3  \cdot 0.5 = 0.00225$
\item<3-> $P(X = 0) = 0.9 \cdot 0.85  \cdot 0.7  \cdot 0.5 = 0.26775$
\item<4-> $P(X = 1) = P(S_1 \cap S_2^c \cap S_3^c \cap S_4^c) + P(S_1^c \cap S_2 \cap S_3^c \cap S_4^c) + + P(S_1^c \cap S_2^c \cap S_3 \cap S_4^c) + P(S_1^c \cap S_2^c \cap S_3^c \cap S_4) = 0.4595$
\item<5-> $P(X = 2) = 0.23, P(X = 3) = 0.0405$ 
\end{itemize}
\end{frame}





\end{document}